\documentclass[a4paper]{article}
\usepackage{amsmath}
\usepackage{minted}
\usemintedstyle{colorful}
\newcommand{\ub}{\mathbf{u}}
\newcommand{\eb}{\mathbf{e}}
\newcommand{\xb}{\mathbf{x}}
\newcommand{\dub}{\mathbf{du}}
\title{Numerical details for implementing a 1D advection in solver in OpenFD
using SBP high order finite differences}
\begin{document}
\maketitle
In this document we describe some numerical details that are useful for
understanding how the example code works, and for asserting that the
implementation is correct.

\section{Continuous problem}
Consider the 1D advection
\begin{align}
        \begin{aligned}
        u_t + au_x &= 0, 0 \leq x \leq 1, \\
           u(x, 0) &= f(x), \\
           u(0, t) &= g(t).
        \end{aligned}
        \label{advection}
\end{align}
If we restrict attention to $a > 0$ and let $g(t) = f(-at)$ than this PDE
describes a solution $u(x, t)$ that is given by propagating the initial
condition $f(x)$ towards the right at the speed $a$.  That is, the solution is
given by $u(x, t) = f(x - at)$. 

\section{Numerical scheme}
In the script \emph{kernel.py}, we use the kernel generator capabilities of
OpenFD to generate GPU code for the following semi-discrete approximation of the
problem (\ref{advection}):
\begin{align}
        \frac{d\ub}{dt} + aD\ub = -\tau\eb_0(u_0 - g(t)).
        \label{approximation}
\end{align}
In this approximation,  $\ub^T = [u_0, u_1, \ldots u_{n-1}] $ is a
\emph{gridfunction} that defines the numerical solution at each grid point.
Furthermore, $D = H^{-1}Q$ is a \emph{summation-by-parts} (SBP) finite
difference operator that discretizes the first derivative. The boundary
condition $u(0, t) = g(t)$ is implemented by using the
\emph{Simultaneous-Approximation-Term} (SAT) method. This method weakly enforces
the boundary condition, and is accomplished by the term in the right-hand side
of (\ref{approximation}). The SAT term consists of a scaling parameter $\tau$
that is determined for stability, a restriction operator $\eb_0$, and the
boundary condition $u_0 - g(t)$. The scaling parameter is set to $\tau =
H_{00}^{-1}$, which is the reciprocal of the first coefficient in the diagonal
norm matrix $H$. The role of the restriction operator $\eb_0$ is to ensure that
the SAT term only acts on the boundary $x=0$. That is, $\eb_0^T\ub = u_0$.

We also need to perform a discretization in time to complete our numerical
scheme. For this purpose, we have chosen to use a low-storage Runge-Kutta
scheme. This method can roughly be summarized as follows for the advancement
from time $t$ to time $t + \Delta t $. For each stage
$s=1,2,\ldots, n$, do
\begin{align}
        \dub &:= a_s\dub + F(\ub, t + c_s\Delta t) \\
        \ub &:= \ub + \Delta t b_s \dub
\end{align}
In this recipe, $\dub$ is extra, temporary storage that we refer to as the
\emph{rates} of the solution $\ub$. The coefficients $a_s$, $b_s$, $c_s$ are
Runge-Kutta coefficients that are known ahead of time. The function $F(\ub, t)$
is the spatial discretization. That is,
\begin{align}
        F(\ub, t) = -aD\ub -\tau\eb_0(u_0 - g(t)).
\end{align}

\section{Kernel generation}
To prepare this numerical description for kernel generation using OpenFD, we split
(\ref{approximation}) into two parts: (i) computation of the PDE without SAT term,
and (ii) computation of the SAT term. Whenever we want to generate a kernel,
there are essentially three aspects we need to be aware of. These are
\begin{itemize}
        \item Input arguments
        \item Output arguments
        \item Bounds
\end{itemize}
For the spatial part of the PDE
computation, there is only one input argument which is the spatial
discretization of this PDE. There are as many output arguments as input
arguments, and the only output argument is the variable that we wish to assign
to. 

The bounds determine the grid points that we wish to compute for, and depend
on what type of computation to perform. What influences the specification of the
bounds is the mixture of matrix vector computation and convolution computation
present in the derivative operator $D$. For points near the boundary, this
operator applies one-sided stencils that changes from point to point, whereas in
the interior, the same central stencil is repeated at each interior point. To
handle this computational heterogeneity, OpenFD splits the computation into
three separate parts, called regions. The typical scenario is to have a left
and right boundary region, and a large interior region. In order to perform this
splitting into regions, the kernel generator in OpenFD needs to known how many
points that each region uses. The number of boundary points depend on which
operator is used. For example, a traditional, second order, SBP finite
difference operator would only use one boundary point per boundary. Consistent
bounds for this operator, on a grid with $n$ points, would be defined by passing
the following to the kernel generator
\begin{minted}{python}
bounds = Bounds(size=n, left=1, right=1).
\end{minted}

\section{Convergence test}
To test that our implementation is correct, we use the \emph{method of
manufactured solutions} (MMS). This method relies on the construction of an a
priori known solution that is chosen to be smooth, and a modification of the
governing equations and boundary conditions such that this solution is
satisfied. 

In this case, we have chosen the manufactured solution to be
\begin{align}
u^*(x, t) = \sin(k x - \omega t), 
\end{align}
where $k = \omega/a$ is the wave number. This solution satisfies the problem
(\ref{advection}) without any modifications to the PDE. In the solver script
\emph{solver.py}, we initialize the numerical solution $\ub$ by setting it to
$\ub = u^*(\xb, 0)$ and we set the time dependent boundary data, $g(t)$, to $
g(t) = u^*(0, t)$.

Note that we use the supercript $*$ to distinguish the exact, manufactured
solution from the numerical solution.

In the script, \emph{convergence.py}, (%FIXME: write this code) 
we have chosen
to define the error $e$ in the numerical solution as the difference between the
numerical $\ub$ and exact $u^*(\xb,t)$ solution, and we have chosen to measure
this error in the weighted, discrete l2-norm
\begin{align}
        \|e\|_h = \| u - u^* \|_h = \sqrt{e^T H e}.
        \label{error}
\end{align}
In this definition, the norm matrix $H$ is diagonal and is associated with the
SBP operator used to discretize in space. For a traditional, second order, SBP
finite difference operator, the norm of the error is
\begin{align}
        \|e\|_h = \sqrt{\frac{h}{2}e_0^2
        +  h\sum_{i=1}^{n-2}e_i^2  + \frac{h}{2}e_{n-1}^2},
\nonumber
\end{align}
where $h$ is the grid spacing.
Using this definition of norm of the error, we compute the convergence rate $q$
by subsequently refining the grid by halving the grid spacing:
\begin{align}
q_j = \log_2( \|e\|_{j}/\|e\|_{j+1} ).
\nonumber
\end{align}
In the definition of the convergence rate $q_j$,  $\|e\|_j$ is the norm of the
error on a grid $j$ with grid spacing $h_j = h (1/2)^j$.

\end{document}
